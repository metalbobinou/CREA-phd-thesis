\begin{center}
\begin{LARGE}
Résumé
\end{LARGE}

\bigskip

\textbf{\@title}

\textbf{\textit{\@subtitle}}

\end{center}

\bigskip

%%%%%%%%%%%%%


La crise sanitaire du COVID-19 a particulièrement accéléré le mouvement de numérisation pourtant déjà initié depuis quelques décennies dans l'enseignement supérieur.
De nombreuses activités ont dû être adaptées dans l'urgence, tout particulièrement les réunions entre enseignants, l'évaluation des étudiants et les enseignements.
Ces activités sont des exemples de processus \og \textit{à forte intensité de connaissances} \fg (ou \og \textit{knowledge intensive processes} \fg en anglais) qui partagent des caractéristiques rendant difficile l'intégration du numérique, telles que :
\begin{itemize}
\item l'abondance de connaissances mobilisables, autant de la part des étudiants lors de leurs travaux que de la part des enseignants évaluant ou adaptant leurs cours,
\item la collaboration entre toutes les parties prenantes du monde de l'enseignement supérieur,
\item la créativité requise pour s'adapter au contexte incertain.
\end{itemize}
Ce besoin rapide de déployer de nouveaux processus à forte intensité de connaissances, ou d'adapter ceux qui existent, se confronte à de nombreux défis connus de ce domaine de recherche spécifique.
La question est de savoir comment réutiliser des connaissances existantes, par exemple des connaissances entreposées en ligne dont l'abondance rend difficile la sélection des plus adaptées aux besoins des enseignants.

Dans cette thèse, nous proposons la méthode CREA réutilisant des cas passés dans le domaine de l'enseignement supérieur, en particulier pour la construction de cours.
La méthode CREA permet de réutiliser des supports de cours existants pour tout d'abord représenter visuellement l'écart entre eux, mais également de proposer des séances de cours présentées sous forme de regroupements de sujets majeurs à aborder.
D'autres types de documents peuvent également être intégrés parmi les supports de cours (des pages webs, ou des articles de recherche), afin de proposer des regroupements adaptés à un public particulier, voire de proposer des regroupements à l'état de l'art de la recherche.
Cette méthode s'appuie sur des outils de traitement automatique de la langue pour extraire les termes employés indépendamment de la langue d'origine, puis sur l'analyse de concepts formels pour calculer des métriques permettant de construire des regroupements de termes et évaluer la similarité des cours fournis en entrée.
Nous proposons également des résultats préliminaires d'une méthode d'ordonnancement des séances.


%%%%%%%%%%%%%

\bigskip

\textbf{Mots clés :
Processus à forte intensité de connaissances,
Adaptive case management,
Gestion de cas,
Réutilisation de connaissances,
Extraction de connaissances,
Analyse de concepts formels,
Traitement automatique du langage,
Numérisation de l'enseignement,
Enseignement}

%%%%%%%%%%%%%%%%%%%%%%%%%%%%%%%%%%%%%%%%%%%%%%%%%%%%%%%%%%%%%%%%%

\newpage

%%%%%%%%%%%%%%%%%%%%%%%%%%%%%%%%%%%%%%%%%%%%%%%%%%%%%%%%%%%%%%%%%

\shorthandoff{:} % REMOVE SPACE BEFORE SEMICOLON

\begin{center}
\begin{LARGE}
Abstract
\end{LARGE}

\bigskip

%CREA : Méthode d’analyse, d’adaptation et de réutilisation des instances de processus à forte intensité de connaissances
%\textbf{\@title}
\textbf{CREA: Method for knowledge intensive process analysis, adaptation and reuse}

% cas d’utilisation dans l’enseignement supérieur en informatique
%\textbf{\textit{\@subtitle}}
\textbf{\textit{use case in postgraduate computer science studies}}

\end{center}

\bigskip

%%%%%%%%%%%%%


Similarly to business domains, digitalisation and virtualisation of processes in higher education started a few decades ago.
During COVID-19 pandemics, the interest in solutions for developing and delivering classes on-line grew substantially.
Nevertheless, solutions allowing for efficient adaptation and reuse of the existing courses and teaching materials in the new circumstances are still lagging.
Transformation of a traditional course to a virtual one, developing a new course adapted for the audience are some examples of  " \textit{knowledge intensive processes} ".
These processes share common properties making digitalization hard, to cite some:
\begin{itemize}
\item they depend on the extensive knowledge and experience of teachers analysing the context and adapting class accordingly,
\item they involve an intense collaboration between all stakeholders in the higher education environment,
\item they require creativity for adapting to uncertain context.
\end{itemize}
Deploying new knowledge intensive processes, or adapting existing ones in this unexpected situation, faced multiple challenges already known from this specific research domain.
The main issue is: how to best reuse existing knowledge, including online stored knowledge, where the abundance of sources and data makes it difficult to select the most suited to teachers' requirements.

In this thesis we propose the CREA method that enables a reuse of past cases in the higher education domain, particularly in courses preparation.
The CREA method supports the reuse of existing courses materials: (i) it presents graphically the gap (i.e. semantic difference) between the courses and (ii) it summarises the class sessions in a form of clusters of main terms to discuss.
Other types of documents can also be integrated within the input courses materials (including web pages, or research articles) in order to propose adapted clusters for specific audiences, or even state-of-the-art materials.
This method relies on natural language processing tools in order to extract the terms regardless of the input language, then it uses formal concept analysis for computing metrics to build clusters of terms and assess the similarity of input materials.
We also propose some preliminary results of a session scheduling method.


%%%%%%%%%%%%%

\bigskip

\textbf{Keywords:
Knowledge intensive process,
Adaptive case management,
Case Management,
Knowledge Reuse,
Knowledge Extraction,
Formal concept analysis,
Natural language processing,
Digitalisation of education,
Education}

\shorthandon{:} % RE-ACTIVATE SPACE BEFORE A SEMICOLON
