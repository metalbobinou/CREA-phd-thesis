\begin{center}
\begin{LARGE}
Remerciements
\end{LARGE}
\end{center}

\bigskip

%%%%%%%%%%%%%

\begin{small}

Je remercie tout d'abord mes encadrantes, Bénédicte Le Grand et Irina Rychkova de m'avoir encadré et accompagné pour cette thèse pendant laquelle beaucoup trop d'évènements imprévus se sont produits (mais au moins, on peut littéralement appeler ça l'aventure de la thèse !).
Beaucoup de chemins, détours, impasses, raccourcis, ... ont été explorés, pour finalement construire cette carte.
Ce fut complexe, long, parfois très laborieux, mais nous y sommes arrivés.
Lorsque j'ai lu pour la première fois l'intitulé du sujet (\og \textit{Exploration des higraphs et de l'analyse formelle de concepts pour la modélisation des processus métier à forte intensité de connaissance et pour l'évaluation des modèles} \fg), je ne voyais absolument pas où commençait et finissait chaque terme... mais je suis aujourd'hui ravi de pouvoir expliquer plusieurs d'entre eux !
Même chose sur les méthodologies de recherche : design science, action research, behavior, et bien d'autres qu'il me reste à découvrir.
Je ne pensais absolument pas toucher à autant de domaines en démarrant une thèse en informatique, mais je suis comblé par toutes ces connaissances (et je souhaite réellement continuer à apprendre et comprendre).
Merci encore Bénédicte et Irina !

\smallskip

Je tiens également à remercier les rapporteurs qui ont accepté de relire mon manuscrit : Alain Wegmann et Guillaume Cleuziou.
Leurs précieux conseils et remarques m'ont effectivement aidé à développer ce qu'il manquait, et j'espère pouvoir continuer à creuser certaines de leurs pistes dans de futurs travaux.
Je remercie aussi Daniela Grigori et Camille Salinesi d'avoir accepté les rôles d'examinateurs.

\smallskip

Merci énormément au CRI et à l'ensemble de ses membres pour l'ensemble de vos avis critiques et les interminables discussions.
Une pensée émue et pleine de souvenirs à Ali Jaffal, Elena Viorica Epure, Elena Kushnareva, Nourhène Ben Rabah, Afef Awadid, Danillo Sprovieri, Asmaa Achtaich : merci pour vos soutiens.
Une pensée moins émue et beaucoup plus actuelle à David Beserra, Floriane Owczarek, Angela Patricia Villota Gomez, Luisa Fernanda Rincon, Sabrine Edded, Houssem Chemingui, et aux tous derniers (qui vont y arriver) : Nicolas Six, Claudia Negri, Ramona Elally, Camilo Correa.
Sans oublier les collègues : Stéphane, Emmanuelle, Corinne, Astrid, Diem, Gabriel, Sarah, Julien, et Brigitte.
Et les anciens camarades de classe : Gözde Kiraç, Nassima Mazouz, Wassila Bey Zekkoub.

\smallskip

Je n'oublie pas les membres du LSE à EPITA pour m'avoir accueilli sur la fin de cette thèse : Marc Espie (les maths, OpenBSD, ... je continue d'apprendre grâce à toi !), Robert Erra (les métriques : nous allons les traiter), Mark Angousturès (on ne lâche pas le data science, ni la recherche), Reda Dehak, Alizée Pénel, Laurent \& Loïca (bip !), et Alexandre Letois.

%\medskip
\smallskip

Plus largement, je remercie \textit{infiniment} l'ensemble des enseignantes et enseignants qui m'ont fait adorer les études et les diverses matières qui existent.
C'est-à-dire, les professeurs des écoles en maternelle et en primaire, les professeurs des collèges et lycées, les professeurs et enseignants[-chercheurs] titulaires ou non à l'EPITA, à l'UQAC, et évidemment à l'Université Paris 1 Panthéon - Sorbonne.
Je garderai éternellement d'excellents souvenirs de vous toutes et tous :
% Maternelle
Laurence, Cécile,
% Primaire
André Ménini, Maria Raison, \'Elisabeth Aubin,
% 6e (Collège)
M\up{me} Croué (même si c'était difficile de retenir toutes les définitions par c\oe{}ur), M\up{me} Oeuvrard, M\up{me} Corbic,
% 5e
M. Brassard, Bernard Roquejoffre, M\up{me} Gautry,
% 4e
Gaston Époté Myke, M. Huchon, Peggy Sultan, M\up{me} Mouchez, M\up{me} Lorgeril,
% 3e
M\up{me} Le Texier, M\up{me} Luccioni, M. Azam, M. Tobaty,
% 2nde (Lycée)
M. Nefati,
% 1ere S
Michel Volkovitch (et ses fabuleuses phrases à traduire concernant les flûtes et les synthétiseurs), M. Garcia, M. Oucif (Pierre Loti, \og Vers Ispahan \fg, \og Aziyadé \fg), M. Begis (pour sa patience infinie),
% Terminale S
M\up{me} Staszak (surtout le tout dernier cours de philosophie le 6 juin 2006), M\up{me} Ehanno,
% EPITA - Prépa
Nathalie "Junior" Bouquet, Christophe "Krisboul" Boullay, Christelle Trémoulet, Anne-Sophie Dujardin, Olivier Rodot, Marwan Burelle, mes ACD Nicolas Ballas (ballas\_n) et Mathieu Sabarly (sabarl\_m),
% EPITA - Ing1
Didier Verna, Akim Demaille, Thierry Géraud, % \LaTeX passe pas avec Verna T.T
% EPITA - SRS
Sébastien Bombal, \'Eric Gaillard, Aurélien Borde, Julien Sterckeman,
% UQAC
M. Bouchard,
% Paris 1
Samira Si-said Cherfi, Saïd Assar, Fayçal Hamdi, Camille Salinesi, Rébecca Deneckère, Irina Rychkova, Bénédicte Le Grand.

Cette thèse a beau être la somme des travaux d'une équipe de chercheurs, sans vous toutes et tous pour m'enseigner autant de choses, je n'aurais jamais acquis assez de connaissances pour mener à bien ce projet de recherche.
Une pensée également pour André Rossano qui m'a particulièrement touché lorsqu'il nous expliquait le mainframe, et ses anecdotes de travail d'une autre ère sur S/370 et l'ASR 33.

%% 1988 - 1989
%% 1989 - 1990
%% 1990 - 1991
%% 1991 - 1992 : Petite Section Maternelle (Jean-François, Laurence)
%% 1992 - 1993 : Moyenne Section Maternelle (Cécile)
%% 1993 - 1994 : Grande Section Maternelle (Catherine, Brigitte)
%% 1994 - 1995 : CP (André Ménini)
%% 1995 - 1996 : CE1 (André Ménini)
%% 1996 - 1997 : CE2 (Maria Raison)
%% 1997 - 1998 : CM1 (Odile ???)
%% 1998 - 1999 : CM2 (Elisabeth Aubin, Madame Debune)
%% 1999 - 2000 : 6e3 (anglais : Mme Pastor, maths : Mme Croué, français : Mme Oeuvrard, histoire/geo : Mme Gallo, svt : Mme Corbic, techno : Mme Dubucq, musique : Mme Fleury, arts plastiques : M ???, EPS : Mme Le Strat)
%% 2000 - 2001 : 5e1 (français : Mme Moumaneix, maths : M Quillivic, histoire/geo : Mme Felip, anglais : Mme Fiévet, svt : Mme Giraudeau, physique/chimie : M Durand / M Garcin, techno : Mme Gautry, musique : M Brassard, arts plastiques : Bernard Roquejoffre, EPS : Mme Le Strat, Latin : M Uda)
%% 2001 - 2002 : 4e5 (physique/chimie : Gaston Epoté Myke, maths : Mme Lorgeril, français : Mme Mouchez, histoire/geo : M Huchon, anglais : Mme Peggy Sultan, svt : Mme , techno : M Marniche, espagnol : Mme Gérinard, musique : xxx, arts plastiques : Mme Trouillard, EPS : Mme "STGEPS" ! Mme Fauvet)
%% 2002 - 2003 : 3e3 (français : Mme Luccioni, maths : Mme Le Texier, histoire/geo : M Tobaty/M Mériau, anglais : Mme Brugerolle, svt : Mme Corbic, physique/chimie : Mme Lorans, techno : M Azam (fabrice), espagnol : Mme Rodriguez, musique : xxx, arts plastiques : Bernard Roquejoffre)
%% 2003 - 2004 : Seconde 4 ~ (physique/chimie : M Nefati (Heddi Neffati ?), Histoire/Géo : Mme Gernigon, svt : Mme Fereira, maths : Mme Graff, français : Mme Gallier, anglais : Mme Bondarenco, espagnol : Mme Vecchini, MPI : Mme Ehanno, EPS : M Vergniol, LCE : Mme Boutin)
%% 2004 - 2005 : Premiere S 4 (svt : Mme Brelet (Anne Brelet ?), Physique/Chimie : M Nefati (Heddi Neffati ?), maths : Mme Filippi, français : Georges Oucif, histoire/geo : M Garcia, anglais : Michel Volkovitch, espagnol : Mme Pays && M Achouchi, EPS : M Begis, LCE : Mme Hadida)
%% 2005 - 2006 : Terminale S 2 (svt : M Vasseur, Physique/Chimie : Mme Ehanno, maths normale : Mme Hubert, maths spécialité : Mme Vallaud, philosophie : Mme Staszak, histoire/geo : M Lucas, anglais : Mme Petit, espagnol : Mme Souche, EPS : M Begis, LCE : M Gille)
%% 2006 - 2007 : InfoSup A2 (ACD : ballas_n Ballas Nicolas & sabarl_m Sabarly Mathieu)
%% 2007 - 2008 : InfoSpe A1
%% 2008 - 2009 : InfoSpe A1
%% 2009 - 2010 : Ing1 Groupe B
%% 2010 - 2011 : Ing2 SRS (Sébastien Bomabal SSI - Aurélien Borde WSEC - Julien Sterckeman Sécu - Jean-François Lubrano AD - Eric Gaillard Réseau)
%% 2011 - 2012 : Ing3 SRS (André Rossano Formateur IBM depuis OS/370)
%% 2012 - 2013 : Aubay (Travail)
%% 2013 - 2014 : Aubay (Travail)
%% 2014 - 2015 : Master SID (Paris 1)
%% 2015 - 2016 : (Paris 1, recherche libre Architecture d'Entreprise & Décentralisation)
%% 2016 - 2017 : Doctorat 1ère année (Contrat Doctoral - Paris 1) : Statechart, HiGraph, & KiP
%% 2017 - 2018 : Doctorat 2e année (Contrat Doctoral - Paris 1) : Ontologies, BPM, EPC
%% 2018 - 2019 : Doctorat 3e année (Contrat Doctoral - Paris 1) : BPM, KIP  ==> Sujet OK ! (+ Passage au BAIP & SCUIO)
%% 2019 - 2020 : Doctorat 4e année (demi-ATER - Paris 1) : BPM, KIP, ACF, TAL/NLP, Règles d'Association, Clustering
%% 2020 - 2021 : Doctorat 5e année (demi-ATER - Paris 1) : BPM, KIP, ACF, TAL/NLP, Clustering
%% 2021 - 2022 : Fin Doctorat (Paris 1 && Enseignant-Chercheur EPITA au LSE)
%%


%\medskip
\smallskip

Je pense également à vous toutes et tous... Jason Brillante en premier pour absolument tout, y compris les moments au Liberty Rock Studio gravés à jamais dans nos c\oe{}urs et mémoires.
Éva Debray pour ses conseils, relectures, encouragements, appels, jus de citrons, explications, remises en question de mes certitudes, ...
Si j'ai appris les méthodes de recherche en informatique au labo, j'ai appris beaucoup de choses sur la philosophie, le contrôle social, Spinoza, les sciences de l'éducation, l'histoire de l'école en France, ... grâce à toi, tes explications, et tes corrections.
Merci Éva.
Soutiens indéfectibles depuis plus de 10 ans maintenant (merci de me remonter le moral) : Geoffrey Tan, Laurent Garcin, Shanand Seeram, Jérémy Meng, Riyad Yakine.
Merci à Kathleen Ripert pour le soutien un peu trop lointain depuis l'Allemagne... mais le soutien est là, et c'est le plus important ! (Tu reviens quand à Paris ?)
Une pensée particulière à Ghislain Guillot (oui, tu comptes beaucoup quand même).

\smallskip

Je remercie toutes et tous les amis et potes du monde entier (GameSeries, Red-Network, Dimension MMX, D.Folk, ...) : Merlin, Vadim, Diana, Renan, Rémi, Frédéric, Justin, Christian, Marion, Jean-Baptiste, Lisa, Guillaume, Sophie, Éléonore, Samantha, Angélina, Johanna, Clara, Marine, Sarah, Elisa, Axel, Stacy, Alex(andra), Axel, Julie, Lionel, Patricia, ... sans oublier Robert Rafie et les épi-potes ! (Raph', Cédric, Pierre, Justin, Fabien, Cyril, Alexandre, Aurélie, Gabriel, Lucas, François, Alban, Florent, Robin, Joe, Nils, ...)

%\medskip
\smallskip

Alexandre Abraham, Lydie Gustafsson, et Egle Tomasi m'ont permis de sauter le pas et réellement faire une thèse : ce manuscrit n'existerait pas sans vous, ni mon changement radical de carrière.

\smallskip

Grâce aux doctorantes et doctorants de Paris 1 (parfois avec le doctorat) et d'ailleurs, j'ai découvert les SHS.
Et sans vous, je serais encore un nigaud qui dirait des bêtises sur vos domaines : Hélène Bénistand, Léon Guillot, Armand Desprairies, Guillaume Noblet, Orianne Tercerie, Juliette Fontaine, Bastien Rueff, Cécile Bourgade, Maëlle de Seze, Milan Bonté, Hugo Vidon, Justine Audebrand, Anaïs Bonano, Sahra Rausch, Karim Abou-Merhi, Evélia Mayenga, Marine Lassery, Matthieu Febvre-Issaly, Typhaine Rahault, Guillemette Prevot, Michaël Pierrelée, Nicolas Jouvin, Clément, ...
Et merci aux étudiantes et étudiants qui m'ont beaucoup appris ou avec qui j'ai bien ri : Joachim "Jojo" Loysel, Sami, Émilie, Thib', Hasna, Lucas, Maximilien, Trystan, Emilio, Jo, Marie, Elsa, Lorenz, Jaspal, Ulysse, Adèle, Diu, Nurlan, ...

Plus généralement, je remercie tou·te·s les camarades de Mobdoc et autres organisations.
Vivement le 27 Nivôse (Zinc) de l'an CCXXX (230) !

%\medskip
\smallskip

Je n'oublie pas les plus importants : mes parents (Robert Boissier et Patricia Kissling), mes grands-parents (Willy, Augusta, Rosa), tous les autres membres (Yvette, Frédérique, Bertrand, Isabelle, Tiffany, Ornella, Robin, ...).
Sans ma mère, je ne sais pas si j'aurais eu le bac (et je rigolerais probablement moins).
Sans mon père, je ne sais pas si j'aurais fait de l'informatique (et je m'intéresserais probablement à beaucoup moins de choses).

\end{small}
